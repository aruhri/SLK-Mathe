\documentclass[12pt,a4paper,twoside,fleqn]{article}
\usepackage[utf8]{inputenc}
\usepackage[german]{babel}
\usepackage[left=25mm, top=20mm, bottom=20mm, right=20mm]{geometry}
\usepackage{graphicx,color,marvosym,upgreek}
\usepackage{amssymb,multicol,overpic}
\usepackage{amsmath}
%\usepackage{ifthen}

\begin{document}

\renewcommand{\thepage}{Seite~\arabic{page}}
% \renewcommand{\thesection}{1.}
% \renewcommand{\thesubsection}{\arabic{subsection}}
\renewcommand{\baselinestretch}{1.2}

\renewcommand{\labelenumi}{{\bf\arabic{enumi}.)}}
\renewcommand{\labelenumii}{{\bf\alph{enumii})}}
\renewcommand{\labelenumiii}{{\bf\roman{enumiii})}}

\newcounter{column}
\renewcommand{\thecolumn}{{\bf\alph{column}\ }}
\newcommand{\labelcolumn}{{\bf\alph{column})\ \ \ }}
\setlength{\itemsep}{0pt}
\setlength{\mathindent}{0cm}
\newcounter{last}


\pagestyle{myheadings}
\markboth{\hfill Kurs: Orthogonalität und Produkte von Vektoren}%
{Kurs: Orthogonalität und Produkte von Vektoren\hfill}
\title{Orthogonalität und Produkte von Vektoren.\\\large{Ein Kurs
  zum selbständigen Lernen.}}
\author{Alexander Ruhri\\
  \small\texttt{a.ruhri@widarschule.de}\\
  \small Aktuelle Version unter \texttt{http://www.ruhri.net/}}
\date{\small Version 0.2, Januar 2012}
% Todo: Lösungen
\maketitle
\section*{Vorbemerkungen}
Bitte beachten Sie beim Bearbeiten dieser Blätter, dass an geeigneten
Stellen Zusammenfassungen mit der ganzen Klasse durchgeführt
werden. Bitte machen Sie Ihren Lehrer darauf aufmerksam, wenn Sie das
Bedürfnis nach einer solchen Besprechung haben. 
\subsection*{Voraussetzungen}
Um mit den Inhalten des Kurses zurechtzukommen ist es notwendig,
dass Sie wissen, was ein Vektor ist, wie man die Länge eines Vektors
berechnet und die Parameterdarstellung von Geraden bereits gelernt
haben. Außerdem sollten Sie die Winkelfunktionen ($\sin, \cos$)
wiederholt haben. 
\tableofcontents
\newpage
\section{Orthogonale Vektoren}
Das Wort orthogonal leitet sich vom griechischen
$o\rho\theta o\varsigma$  (orthos)
„richtig, recht-“ und $\gamma\omega\nu\iota\alpha$ (gonia) „Ecke, Winkel“ ab
(vergl. \texttt{http://de.wikipedia.org/wiki/Orthogonalität}) und
bedeutet „rechtwinklig“. Wenn zwei Vektoren zueinander orthogonal
sind, kann man das in Zeichen so schreiben: $$\vec{a}\perp\vec{b}$$
\subsection{Orthogonale Vektoren finden}
\begin{enumerate}
\item Zeichnen Sie Pfeile der Vektoren in ein Koordinatensystem ein und
  finden Sie (gleich lange) zu ihnen orthogonale Vektoren. Zeichnen
  Sie die orthogonalen Vektoren ein und geben Sie ihre Koordinaten
  an. 

  \begin{multicols}{4}
    \begin{enumerate}
    \item $\vec{a}=  
      \begin{pmatrix}
        0\\1
      \end{pmatrix}
      $
    \item $\vec{b}=  
      \begin{pmatrix}
        1\\1
      \end{pmatrix}
      $
    \item $\vec{c}=  
      \begin{pmatrix}
        3\\4
      \end{pmatrix}
      $
    \item $\vec{d}=  
      \begin{pmatrix}
        -7\\2
      \end{pmatrix}
      $
    \end{enumerate}
  \end{multicols}
\item Wenn Sie eine einfache Regel gefunden haben, mit der man gleich lange
  orthogonale Vektoren aus den Koordinaten des gegebenen Vektors bilden
  kann (im $\mathbb{R}^2$), dann formulieren und notieren Sie diese
  Regel. Ansonsten  versuchen Sie anhand weiterer Beispiele die Regel zu
  finden.
\item Beschreiben Sie das Gebilde, das alle zu einem Vektor (wieder im
  $\mathbb{R}^2$)  orthogonalen Vektoren (aller Längen), bilden.
\item Überlegen Sie, wie dieses Gebilde aller zu einem Vektor
  orthogonaler Vektoren im $\mathbb{R}^3$ aussieht. Wie viele gleich
  lange orthogonale Vektoren hat ein Vektor im $\mathbb{R}^3$?
\end{enumerate}

\subsection{Vektoren auf Orthogonalität prüfen}
Als Test auf Orthogonalität zweier Vektoren kann man den Satz des
Pythagoras verwenden, denn er gilt nur in orthogonalen Dreiecken.
\vspace{.2cm}

\begin{minipage}[t]{.3\linewidth}
  \begin{overpic}[scale=.5,tics=10]%
    {pics/Dreieck}
    \put(-3,25){\scriptsize $\vec{b}$} \put(45,-3){\scriptsize
      $\vec{a}$} \put(45,30){\scriptsize $\vec{b}-\vec{a}$}
  \end{overpic}

\end{minipage}\hfill
\begin{minipage}[b]{.65\linewidth}
  Die beiden Vektoren in nebenstehender Skizze sind also genau dann
  orthogonal zueinander, wenn gilt: 
  $$|\vec{a}|^2 + |\vec{b}|^2 = |\vec{b} - \vec{a}|^2.$$
\end{minipage}
Für die rechte Seite gilt: $$|\vec{b} - \vec{a}|=\sqrt{(b_1-a_1)^2 + (b_2-a_2)^2}$$
Also: $$ |\vec{b} - \vec{a}|^2 =(b_1-a_1)^2 + (b_2-a_2)^2 =
(b_1^2-2b_1a_1 + a_1^2) + (b_2^2-2b_2a_2 + a_2^2)$$
Durch Ausklammern erhalten wir:
$$ |\vec{b} - \vec{a}|^2 = (a_1^2+ a_2^2)+ (b_1^2+b_2^2)
-2\cdot(a_1b_1+a_2b_2)$$
Für die linke Seite erhalten wir auf gleiche Weise: 
$$|\vec{a}|^2 + |\vec{b}|^2 = (a_1^2+a_2^2) + (b_1^2+b_2^2)$$
Wenn wir linke und rechte Seite vergleichen, merken wir, dass der Satz
des Pythagoras nur dann gilt wenn $ 2\cdot(a_1b_1+a_2b_2)=0$ also wenn
gilt: 
$$a_1b_1+a_2b_2=0$$
Insgesamt gilt also: $$\vec{a}\perp\vec{b} \Leftrightarrow a_1b_1+a_2b_2=0$$
\begin{enumerate}
\item Überprüfen Sie die von Ihnen gefundenen Vektoren mit dieser
  Regel auf Orthogonalität.
\item Überprüfen Sie folgende Paare von Vektoren auf Orthogonalität:
\begin{multicols}{4}
    \begin{enumerate}
    \item $ 
      \begin{pmatrix}
        2\\1
      \end{pmatrix};
      \begin{pmatrix}
        -6\\12
      \end{pmatrix}
      $
    \item $
      \begin{pmatrix}
        14\\-8
      \end{pmatrix} ;
      \begin{pmatrix}
        4\\7
      \end{pmatrix}
      $
    \item $
      \begin{pmatrix}
        -1\\-4
      \end{pmatrix} ;
      \begin{pmatrix}
        4\\1
      \end{pmatrix}
      $
    \item $
      \begin{pmatrix}
        7\\2
      \end{pmatrix};
      \begin{pmatrix}
        1\\-3,5
      \end{pmatrix}
      $
    \end{enumerate}
  \end{multicols}
\item Formulieren Sie diese Regel für den $\mathbb{R}^3$ indem Sie 
  $\vec{a}= \begin{pmatrix}
        a_1\\a_2\\a_3
      \end{pmatrix}$ und 
  $\vec{b}= \begin{pmatrix}
        b_1\\b_2\\b_3
      \end{pmatrix}$
verwenden.
\item Überprüfen Sie folgende Paare von dreidimensionalen Vektoren auf
  Orthogonalität:
\begin{multicols}{4}
    \begin{enumerate}
    \item $ 
      \begin{pmatrix}
        1\\0\\1
      \end{pmatrix};
      \begin{pmatrix}
        0\\1\\0
      \end{pmatrix}
      $
    \item $
      \begin{pmatrix}
        2\\3\\4
      \end{pmatrix} ;
      \begin{pmatrix}
        -4\\0\\2
      \end{pmatrix}
      $
    \item $
      \begin{pmatrix}
        1\\2\\3
      \end{pmatrix} ;
      \begin{pmatrix}
        2\\2\\-2
      \end{pmatrix}
      $
    \item $
      \begin{pmatrix}
        0\\0\\2
      \end{pmatrix};
      \begin{pmatrix}
        0\\-1\\1
      \end{pmatrix}
      $
    \end{enumerate}
  \end{multicols}
\item Finden Sie $a\in\mathbb{R}$, so dass die Vektoren orthogonal zueinander stehen.
  \begin{multicols}{4}
    \begin{enumerate}
    \item $ 
      \begin{pmatrix}
        3\\4\\1
      \end{pmatrix};
      \begin{pmatrix}
        -1\\2\\a
      \end{pmatrix}
      $
    \item $
      \begin{pmatrix}
        7\\1\\5
      \end{pmatrix} ;
      \begin{pmatrix}
        1\\a\\2
      \end{pmatrix}
      $
    \item $
      \begin{pmatrix}
        10\\27\\0,5
      \end{pmatrix} ;
      \begin{pmatrix}
        a\\1\\-14
      \end{pmatrix}
      $
    \item $
      \begin{pmatrix}
        8\\3\\1
      \end{pmatrix};
      \begin{pmatrix}
        2\\a\\a
      \end{pmatrix}
      $
    \end{enumerate}
  \end{multicols}
\end{enumerate}\newpage
\section{Skalarprodukt}
Die verwendete Rechenart heißt das {\em Skalarprodukt} zweier
Vektoren und es gilt:
$$ \vec{a} \cdot \vec{b} =
\begin{pmatrix}
        a_1\\a_2\\a_3
      \end{pmatrix} \cdot
      \begin{pmatrix}
        b_1\\b_2\\b_3
      \end{pmatrix} = 
      a_1b_1 + a_2b_2+a_3b_3$$
Der Name Skalarprodukt weist darauf hin, dass das Ergebnis des
Produktes ein {\em Skalar}, eine Maßzahl ist. Das Skalarprodukt wird
{\em immer} mit einem Punkt geschrieben.
Mit dem Ergebnis von oben gilt also:
$$\vec{a}\cdot\vec{b} = 0 \Leftrightarrow \vec{a}\perp\vec{b}$$


\subsection{Skalarprodukt und Orthogonalität}
\begin{enumerate}
\item Berechnen Sie das Skalarprodukt der beiden Vektoren.
  \begin{multicols}{4}
    \begin{enumerate}
    \item  $\begin{pmatrix}
      5\\0
    \end{pmatrix};
   \begin{pmatrix}
      1\\3
    \end{pmatrix} $
  \item $\begin{pmatrix}
      1\\3\\1
    \end{pmatrix};
   \begin{pmatrix}
      2\\5\\1
    \end{pmatrix} $
  \item $\begin{pmatrix}
      1\\3\\5
    \end{pmatrix};
   \begin{pmatrix}
      5\\3\\1
    \end{pmatrix} $
  \item  $\begin{pmatrix}
      -11\\4\\1
    \end{pmatrix};
   \begin{pmatrix}
      1\\2\\3
    \end{pmatrix} $
    \end{enumerate}
  \end{multicols}
\item Überprüfen Sie, ob die sich schneidenden Geraden $g$ und $h$
  zueinander orthogonal sind ($s\in\mathbb{R}$).
  
$$g: \vec{x}= 
    \begin{pmatrix}
      2\\-2\\0
    \end{pmatrix} 
    + s\cdot
    \begin{pmatrix}
      -5\\1\\0
    \end{pmatrix};\quad
    h: \vec{x}= 
    \begin{pmatrix}
      5\\-1\\0
    \end{pmatrix} 
    + s\cdot
    \begin{pmatrix}
      -2\\2\\0
    \end{pmatrix}$$

\item Geben Sie eine Parameterdarstellung einer Geraden $h$ an, die
  die Gerade $g$ orthogonal schneidet($s\in\mathbb{R}$).

\begin{multicols}{2}
  \begin{enumerate}
  \item $g: \vec{x}=   
    \begin{pmatrix}
      3\\3\\7
    \end{pmatrix} 
    + s\cdot
    \begin{pmatrix}
      7\\17\\-2
    \end{pmatrix}$
  \item $g: \vec{x}=   
    \begin{pmatrix}
      -1\\11\\-1
    \end{pmatrix} 
    + s\cdot
    \begin{pmatrix}
      1\\2\\-3
    \end{pmatrix}$ 
  \end{enumerate}
\end{multicols}
\item Untersuchen Sie, ob das Dreieck $ABC$ rechtwinklig ist.
   \begin{enumerate}
  \item $A(3|4|-8);\; B(6|5|-4) ;\; C(5|2|-9)$
  \item $A(-1|6|-10);\; B(0|3|-11) ;\; C(-5|0|-7)$
  \end{enumerate}
\item Bestimmen Sie {\em alle} Vektoren die sowohl zu $\vec{a}$ also
  auch zu $\vec{b}$ orthogonal sind.
  \begin{multicols}{2}
    \begin{enumerate}
    \item $\vec{a}=   
    \begin{pmatrix}
      1\\2\\3
    \end{pmatrix};
    \vec{b}= 
    \begin{pmatrix}
      2\\0\\3
    \end{pmatrix}$
  \item $\vec{a}=   
    \begin{pmatrix}
      2\\3\\-1
    \end{pmatrix};
    \vec{b}= 
    \begin{pmatrix}
      5\\-1\\-2
    \end{pmatrix}$
    \end{enumerate}
  \end{multicols}
\item Überprüfen Sie, ob das Viereck $ABCD$ ein Rechteck ist.
  \begin{enumerate}
  \item $A(0|4|3);\; B(-1|2|2);\; C(-12|3|11);\; D(-11|5|12)$
  \item $A(13|-4|-5);\; B(12|-5|2);\; C(7|3|1);\; D(8|4|4)$
  \end{enumerate}
\item Drücken Sie die Diagonalen des Vierecks $ABCD$ mit $A(-2|-2)$,
  $B(0|3)$, $C(3|3)$ und $D(3|0)$ durch Vektoren aus. Überprüfen sie
  diese Vektoren auf Orthogonalität.
\end{enumerate}\newpage
\subsection{Winkelberechnung mithilfe des Skalarprodukts}
Der kleinere Winkel zwischen zwei nicht kollinearen Vektoren ist der
Winkel zwischen diesen Vektoren.
Für das Skalarprodukt gilt folgende Beziehung:
$$ \vec{a}\cdot\vec{b}=  |\vec{a}|\cdot|\vec{b}|\cdot \cos{(\alpha)}$$
bzw.
$$\cos{(\alpha)} = \frac{\vec{a}\cdot\vec{b}}{|\vec{a}|\cdot|\vec{b}|}
\textrm{\quad mit } 0^\circ\leq \alpha\leq 180^\circ$$

\begin{enumerate}
\item Überlegen Sie, wie der Satz von der Orthogonalität ($\alpha =
  90^\circ$) mit diesem Satz in Übereinstimmung gebracht werden kann.
\item * Beweisen Sie den Satz mithilfe des Kosinussatzes.
\item Bestimmen Sie die Größe des Winkels zwischen den Vektoren.
  \begin{multicols}{4}
    \begin{enumerate}
    \item  $\begin{pmatrix}
      5\\0
    \end{pmatrix};
   \begin{pmatrix}
      1\\3
    \end{pmatrix} $
  \item $\begin{pmatrix}
      1\\3\\1
    \end{pmatrix};
   \begin{pmatrix}
      2\\5\\1
    \end{pmatrix} $
  \item $\begin{pmatrix}
      1\\3\\5
    \end{pmatrix};
   \begin{pmatrix}
      5\\3\\1
    \end{pmatrix} $
  \item  $\begin{pmatrix}
      -11\\4\\1
    \end{pmatrix};
   \begin{pmatrix}
      1\\2\\3
    \end{pmatrix} $
    \end{enumerate}
  \end{multicols}
\item Berechnen Sie die Seitenlängen und Winkel im Dreieck
  $ABC$. Zeichnen Sie für die Teilaufgabe a) und b) das Dreieck und
  messen Sie nach.
  \begin{multicols}{2}
    \begin{enumerate}
    \item $A(2|1); B(5|-1); C(4|3)$
    \item $A(1|1); B(9|-2); C(3|8)$
    \item $A(5|0|4); B(3|0|0); C(5|4|0)$
    \item $A(5|1|5); B(5|5|3); C(3|3|5)$
    \end{enumerate}
  \end{multicols}
\item Der Winkel zwischen den Vektoren is $\alpha$. Bestimmen Sie die
  fehlende Koordinate.
  \begin{multicols}{2}
    \begin{enumerate}
     \item  $\begin{pmatrix}
      0\\1\\0
    \end{pmatrix};
   \begin{pmatrix}
      \sqrt{3}\\b\\0
    \end{pmatrix}; \;
    \alpha = 30^\circ$ 
  \item  $\begin{pmatrix}
      0\\0,5\\0,5
    \end{pmatrix};
   \begin{pmatrix}
      1\\0\\c
    \end{pmatrix};\;
    \alpha = 60^\circ$
    \end{enumerate}
  \end{multicols}
\item Gegeben sind die Vektoren
  $\vec{a}=\begin{pmatrix} 2\\3 \end{pmatrix}$ und 
  $\vec{b}=\begin{pmatrix} -1\\5 \end{pmatrix}$. Bestimmen Sie jeweils
  die Größe des Winkels zwischen $\vec{a}$ und $\vec{b}$, $-\vec{a}$
  und $\vec{b}$, $\vec{a}$ und $-\vec{b}$ sowie $-\vec{a}$ und $-\vec{b}$.
\item Finden Sie heraus, für welche Winkel das Skalarprodukt negativ ist.
\item Für welche Vektoren gilt: $ \vec{a}\cdot\vec{b}=  |\vec{a}|\cdot|\vec{b}|$?
\item Ein Quader ist $8\,$cm lang, $5\,$cm breit und $3\,$cm
  hoch. $A$, $B$, $C$ und $D$ seien die Ecken seiner Grundfläche. $M$
  ist der Schnittpunkt der Raumdiagonalen. Berechnen Sie $\angle AMB$
  und $\angle BMC$. ($\angle ABC$ ist der Winkel mit Scheitelpunkt $B$
  zu den Punkten $A$ und $C$. Der Scheitelpunkt steht immer in der Mitte.)
\end{enumerate}
\newpage

\section{Vektorprodukt}
Es gibt ein weiteres Produkt von Vektoren: das Vektorprodukt. Wie der
Name schon sagt, ist das Ergebnis dieses Produkts wieder ein
Vektor. Das Vektorprodukt existiert nur für 3-dimensionale
Vektoren, es wird immer mit einem Kreuz geschrieben:
$\vec{a}\times\vec{b}$ (lies: „a~Kreuz~b“). Für $ \vec{a} = 
\begin{pmatrix}
  a_1\\a_2\\a_3
\end{pmatrix} \textrm{ und }
\vec{b} = \begin{pmatrix}
  b_1\\b_2\\b_3
\end{pmatrix} $ gilt:
$$\vec{a}\times\vec{b}=
\begin{pmatrix}
  a_1\\a_2\\a_3
\end{pmatrix}
\times
\begin{pmatrix}
  b_1\\b_2\\b_3
\end{pmatrix}
=
\begin{pmatrix}
  a_2b_3-a_3b_2\\
  a_3b_1-a_1b_3\\
  a_1b_2-a_2b_1
\end{pmatrix}
$$

\begin{enumerate}
\item Berechnen Sie jeweils $\vec{a}\times\vec{b}$.
  \begin{multicols}{2}
    \begin{enumerate}
    \item $\vec{a} = 
\begin{pmatrix}
  1\\2\\3
\end{pmatrix}$;
$\vec{b} = \begin{pmatrix}
  4\\5\\6
\end{pmatrix} $
\item \label{vecprod_1}
$\vec{a} = 
\begin{pmatrix}
  1\\2\\0
\end{pmatrix}$;
$\vec{b} = \begin{pmatrix}
  3\\1\\0
\end{pmatrix} $
\item\label{vecprod_2}
  $\vec{a} = 
\begin{pmatrix}
  -4\\3\\0
\end{pmatrix}$;
$\vec{b} = \begin{pmatrix}
  2\\-9\\-1
\end{pmatrix} $
\item  $\vec{a} = 
\begin{pmatrix}
  0\\1\\-3
\end{pmatrix}$;
$\vec{b} = \begin{pmatrix}
  8\\-8\\5
\end{pmatrix} $
    \end{enumerate}
  \end{multicols}
\item Berechnen Sie jeweils mit den Vektoren aus \ref{vecprod_1} und \ref{vecprod_2}.
  \begin{multicols}{3}
    \begin{enumerate}
    \item $\vec{b}\times\vec{a}$
    \item $\vec{a}\cdot \left(\vec{a}\times\vec{b}\right)$
    \item $\vec{b}\cdot \left(\vec{a}\times\vec{b}\right)$
    \end{enumerate}
     \end{multicols}
 Notieren Sie Ihre Schlussfolgerungen!
\item Das Viereck $ABCD$ sei ein Parallelogramm mit den folgenden
  gegebenen Punkten: $A(0|1|-3)$, $B(-1|1|-2)$ und $D(1|3|1)$.
  \begin{enumerate}
  \item Bestimmen Sie die Koordinaten des Punktes $C$.%(0|3|2)
  \item Berechnen Sie die Fläche des Parallelograms.
  \item Berechnen Sie $|\vec{AB}\times\vec{AD}|$.
  \item Notieren Sie Ihre Schlussfolgerung und versuchen Sie anhand
    Ihrer Ergebnisse und Überlegungen eine
    Formel für die Länge des Vektorprodukts zu finden
    ($|\vec{a}\times\vec{b}|=\ldots$).
  \end{enumerate}
\item Berechnen Sie weitere Vektorprodukte zur Übung mit folgenden
  Vektoren:
  $$\begin{pmatrix}
  1\\-1\\-2
\end{pmatrix};
\begin{pmatrix}
  3\\-7\\-11
\end{pmatrix};
\begin{pmatrix}
  \frac 4 7\\5\\-\frac 1 2
\end{pmatrix};
\begin{pmatrix}
  0\\0\\-2
\end{pmatrix};
\begin{pmatrix}
  4\\5\\0
\end{pmatrix};
\begin{pmatrix}
  -3\\2\\0
\end{pmatrix};
\begin{pmatrix}
  0\\1\\-9
\end{pmatrix}$$
\end{enumerate}


\end{document}


