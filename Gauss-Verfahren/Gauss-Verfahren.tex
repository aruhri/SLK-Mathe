\documentclass[12pt,a4paper,twoside,fleqn]{article}
\usepackage[utf8]{inputenc}
\usepackage[german]{babel}
\usepackage[left=25mm, top=20mm, bottom=20mm, right=20mm]{geometry}
\usepackage{graphicx,color,marvosym,upgreek,url}
\usepackage{amssymb,multicol,overpic}
\usepackage{amsmath}
%\usepackage{ifthen}

%% http://www.ctan.org/pkg/exsheets
\usepackage[totoc=false]{exsheets}


\begin{document}

\renewcommand{\thepage}{Seite~\arabic{page}}
% \renewcommand{\thesection}{1.}
% \renewcommand{\thesubsection}{\arabic{subsection}}
\renewcommand{\baselinestretch}{1.2}

\renewcommand{\labelenumi}{{\bf\arabic{enumi}.)}}
\renewcommand{\labelenumii}{{\bf\alph{enumii})}}
\renewcommand{\labelenumiii}{{\bf\roman{enumiii})}}

\newcounter{column}
\renewcommand{\thecolumn}{{\bf\alph{column}\ }}
\newcommand{\labelcolumn}{{\bf\alph{column})\ \ \ }}
\setlength{\itemsep}{0pt}
\setlength{\mathindent}{0cm}
\newcounter{last}

\SetupExSheets{question/headings=runin,question/type=exam}


\pagestyle{myheadings}
\markboth{\hfill Kurs: LGS und Gauß-Verfahren}%
{Kurs: LGS und Gauß-Verfahren\hfill}
\title{Das Gauß-Verfahren zum Lösen von Linearen Gleichungssystemen\\\large{Ein Kurs
  zum selbständigen Lernen.}}
\author{Alexander Ruhri\\
  \small\texttt{a.ruhri@widarschule.de}\\
  \small Aktuelle Version unter \texttt{http://www.ruhri.net/}}
\date{\small Version 0.1, Februar 2017}
% Todo: Lösungen
\maketitle
\section*{Vorbemerkungen}
Bitte beachten Sie beim Bearbeiten dieser Blätter, dass an geeigneten
Stellen Zusammenfassungen mit der ganzen Klasse durchgeführt
werden. Bitte machen Sie Ihren Lehrer darauf aufmerksam, wenn Sie das
Bedürfnis nach einer solchen Besprechung haben. 
\subsection*{Voraussetzungen}
Um mit den Inhalten des Kurses zurechtzukommen ist es notwendig,
dass Sie wissen, wie man lineare Gleichungssysteme mit 2 Unbekannten
mittels Additionsverfahren löst. Außerdem sollten Sie
Äquivalenzumformungen von Gleichungen beherrschen.
\tableofcontents
\newpage
\section{Lineare Gleichungssysteme (LGS)}
Bisher haben Sie gelernt, LGS mit zwei Unbekannten zu lösen. Sie haben
dazu möglicherweise verschiedene Verfahren kennen gelernt:
\begin{itemize}
\item Einsetzverfahren
\item Gleichsetzungsverfahren
\item Additionsverfahren
\item Determinantenverfahren
\end{itemize}

Das Gauß-Verfahren ist eine Anwendung des \emph{Additionsverfahrens}, mit der
man auf sichere Weise LGS mit drei und mehr Variablen lösen kann.
\subsection{Eine kurze Wiederholung}

\begin{question}%
  Lösen Sie die folgenden LGS mit dem Additionsverfahren.
  \begin{multicols}{3}
    \begin{enumerate}
    \item $
      \left|\begin{array}{rcl}
        4x_1+3x_2&=&-5 \\
        2x_1+x_2&=&1
      \end{array}\right|$
    \item 
      $\left|\begin{array}{rcl}
         x_1+5x_2&=&7 \\
         -x_1+x_2&=&-1
       \end{array}\right|$
     \item 
       $\left|\begin{array}{rcl}
          2x_1+x_2&=&2 \\
          3x_1-x_2&=&23
        \end{array}\right|$
      \item 
        $\left|\begin{array}{rcl}
          -4x_1+2x_2&=& 2\\
          3x_1+x_2&=&-1
        \end{array}\right|$
       \item 
        $\left|\begin{array}{rcl}
          -x_1+x_2&=& 2\\
           x_2&=&-2
        \end{array}\right|$
       \item 
        $\left|\begin{array}{rrcl}
          2x_1&+7x_2&=& 9\\
           x_1&&=&4
        \end{array}\right|$
      \end{enumerate}
    \end{multicols}
  \end{question}
  \begin{solution}
    \begin{multicols}{6}
    \begin{enumerate}
    \item $(4;-7)$
    \item $(2;1)$
    \item $(5;-8)$
    \item $(-\frac 2 5 ; \frac 1 5 )$
    \item $(-4;-2)$
    \item $(4;\frac 1 7)$
    \end{enumerate}
  \end{multicols}
\end{solution}
              
Welche dieser Aufgaben konnten Sie besonders leicht lösen? Woran lag
das? Notieren Sie Ihre Überlegungen!
\subsection{Wir machen es ein wenig komplexer}
\begin{question}%
  Lösen Sie die folgenden LGS mit dem Additionsverfahren. Beschreiben
  Sie, welche Eigenschaft der LGS Ihnen das Lösen erleichtert.
  \begin{multicols}{2}
    \begin{enumerate}
    \item  
      $\left|\begin{array}{rcl}
         -x_1+2x_2-x_3&=& 0\\
         3x_2+x_3&=& 9\\
         5x_3&=&15
       \end{array}\right|$
      \item   
        $\left|\begin{array}{rrrcl}
           7x_1&+20x_2&-2x_3&=& -16\\
           -2x_1&& +2x_3&=& 6\\
           3x_1&&&=&-6
         \end{array}\right|$ 
     \item 
        $\left|\begin{array}{rllcl}
           -2x_1&+2x_2&&=& -18\\
           5x_1&+4x_2& +3x_3&=&9 \\
           2x_1&-x_2&&=&13
         \end{array}\right|$ 
     \item 
       $\left|\begin{array}{rlllcl}
                &3x_2&&&=&6    \\
                x_1&+4x_2&&+3x_4&=&3 \\
                -3x_1&+2x_2&+x_3&-x_4&=& 2\\
                &-x_2&&+x_4&=&-4
              \end{array}\right|$ 
    \end{enumerate}
  \end{multicols}
\end{question}
\begin{solution}
  \begin{multicols}{4}
    \begin{enumerate}
    \item $(1;2;3)$
    \item $(-2;0;1)$
    \item $(4;-5;3)$
    \item $(1;2;-1;-2)$
    \end{enumerate}
  \end{multicols}
\end{solution}

\begin{question}
  Wir lassen erst die erleichternde Eigenschaft weg und dann bricht in
  der zweiten Aufgabe auch noch Chaos aus. Versuchen Sie nun, die LGS
  zu lösen. Beschreiben Sie die nötigen Schritte. Sollten Sie hier
  scheitern, gehen Sie zum nächsten Abschnitt und versuchen Sie es mit
  dem neu erworbenen Wissen später nochmal.
  \begin{multicols}{2}
    \begin{enumerate}
    \item   
      $\left|
        \begin{array}{rllcl}
          x_1&+2x_2&-2x_3&=& 5\\
          -x_1&-2x_2&+4x_3&=& 1\\
          -x_1&+3x_2&-5x_3&=&-11
        \end{array}\right|$
    \item 
      $\left|
        \begin{array}{rlcll}
          -2x_3&+2x_2&=& 1 &-x_1\\
          2x_1&-3x_2&=& 2 &-3x_3\\
          3x_2&+3x_3&=&x_1&+5
        \end{array}\right|$
    \end{enumerate}
  \end{multicols}
\end{question}
\begin{solution}
  \begin{multicols}{4}
    \begin{enumerate}
    \item $(5;3;3)$
    \item $(1;1;1)$
    \end{enumerate}
  \end{multicols}
\end{solution}


% \section{Das Gauß-Verfahren}
% Wie Sie wahrscheinlich oben erkannt haben, lassen sich
% Gleichungssysteme dann besonders leicht lösen, wenn es immer eine
% Gleichung mit einer Unbekannten weniger gibt. Wenn man die Gleichungen
% dann umsortiert, ergibt sich die {\em Stufenform des
%   Gleichungssystems}:

% $$\left|\begin{array}{rcl}
%          -x_1+2x_2-x_3&=& 0\\
%          3x_2+x_3&=& 9\\
%          5x_3&=&15
%        \end{array}\right| $$

% Man sieht leicht, dass sich aus der letzten Gleichung $x_3=3$
% ergibt. Das kann dann in die zweite Gleichung eingesetzt
% werden. Daraus ergibt sich dann $x_2=2$. Jetzt können beide gewonnenen
% Lösungen in die erste Gleichung eingesetzt werden und man erhält
% $x_1=1$.

% Damit man die Übersicht behält, ist es zwingend notwendig, dass die
% Variablen aller Gleichungen auf der linken Seite stehen und gleiche
% Variablen immer direkt vertikal geordnet sind. Konstante Terme
% (Zahlen) stehen auf der rechten Seite.

% Carl Friedrich Gauß (1777 -- 1855) hat ein Verfahren beschrieben, mit
% dem man ein beliebiges LGS in die Stufenform bringen kann. Dabei sind
% folgende Umformungen erlaubt:
% \begin{itemize}
% \item das Vertauschen zweier Gleichungen,
% \item das Multiplizieren einer Gleichung mit einer Zahl ungleich null
%   und
% \item die Addition einer Gleichung zu einer anderen.
% \end{itemize}


\subsection{Aufgaben}
\begin{question}%
 Bringen Sie die LGS in Stufenform und bestimmen Sie dann die
 Lösung.
  \begin{multicols}{2}
    \begin{enumerate}
     \item 
      $\left|
        \begin{array}{rlcl}
          x_1&+x_2&=& 5\\
          -4x_1&+3x_2&=&-6
        \end{array}\right|$ 
       \item 
      $\left|
        \begin{array}{rllcl}
          x_1&+x_2&-2x_3&=& -5\\
          -2x_1&-2x_2&+x_3&=& 1\\
          -3x_1&-4x_2&-x_3&=&-2
        \end{array}\right|$ 
    \item 
      $\left|
        \begin{array}{rllcl}
          x_1&+4x_2&-12x_3&=& -15\\
          -7x_1&+2x_2&+9x_3&=& 0\\
          -3x_1&-14x_2&-2x_3&=&9
        \end{array}\right|$ 
     \item 
      $\left|
        \begin{array}{rllcl}
          x_1&-3x_2&-2x_3&=& 25\\
          -3x_1&+7x_2&&=& 0\\
          5x_1&+11x_2&-2x_3&=&1
        \end{array}\right|$ 
     \item 
      $\left|
        \begin{array}{rllcl}
          4x_1&-3x_2&-2x_3&=& 1\\
          -8x_1&+3x_2&+4x_3&=&-1 \\
          x_1&+6x_2&+4x_3&=&\frac 1 4
        \end{array}\right|$ 
      \item 
      $\left|
        \begin{array}{rlllcl}
          x_1&-2x_2&+2x_3&+x_4&=& -1\\
          3x_1&-5x_2&+4x_3&+2x_4&=&-3 \\
          x_1&+\frac 1 2 x_2&+\frac 1 3 x_3&-\frac 1 4 x_4&=& 4\\
          x_1&-x_2&+x_3&+x_4&=&-2
        \end{array}\right|$ 
    \end{enumerate}
  \end{multicols}
\end{question}
\begin{solution}
  \begin{multicols}{4}
    \begin{enumerate}
    \item $(3;2)$
    \item $(5;-4;3)$
    \item $(1;-1;1)$
    \item $(7;-3;\frac 1 2)$
    \item $(\frac 1 4; \frac 1 3;-\frac 1 2)$
    \item $(1;2;3;-4)$
    \end{enumerate}
  \end{multicols}
\end{solution}
\begin{question}%
  Die Quersumme einer 4-stelligen Zahl ist 17. Die Summe aus Zehner-
  und Einerziffer ist gleich der Tausenderziffer. Die Summe aus
  Hunderter- und Zehnerziffer ist gleich der Einerziffer. Die Summe
  der ersten beiden Zifern ist um 3 größer als die Summe der beiden
  hinteren Ziffern.  Bestimmen Sie die gesuchte Zahl.
\end{question}
\begin{solution}
  Wir suchen die 4 Ziffern einer Zahl: $x_1,x_2, x_3, x_4$. Aus den
  Bedingungen ergibt sich folgendes LGS:

    $\left|
        \begin{array}{rlllcl}
          x_1&+x_2&+x_3&+x_4&=& 17\\
          x_3&+x_4&&&=&x_1 \\
          x_2&+x_3&&&=&x_4\\
          x_1&+x_2&&&=&x_3+x_4+3
        \end{array}\right|$ \\
     Die gesuchte Zahl ist 7325.
\end{solution}
\begin{question}
  Ein Becken mit einem Fassungsvermögen von 1000\,l wird durch drei
  Pumpen befüllt. Pumpe 1 und Pumpe 2 benötigen für die Füllung
  zusammen 45 Minuten. Die dritte Pumpe füllt das Becken zusammen mit
  der ersten Pumpe in einer Stunde und 30 Minuten. Die dritte Pumpe
  arbeitet nur halb so schnell wie die zweite. 
  Wie lange benötigt jede
  Pumpe alleine für die Füllung des Beckens? 
  Wie viel Zeit vergeht,
  wenn man das Becken gleichzeitig mit allen drei Pumpen
  befüllt?\footnote{\url{http://www.schule-bw.de/unterricht/faecher/mathematik/3material/sek1/algebra/glsys/}}
\end{question}
\begin{solution}
  Pumpe 1 ist kaputt, Pumpe 2 braucht 45 Minuten um das Becken zu
  füllen, Pumpe 3 braucht 90 Minuten. Alle drei Pumpen zusammen
  brauchen eine halbe Stunde.
\end{solution}
\begin{question}
  In einem Stall sind Hühner, Schafe, Ziegen und Spinnen.
  Die Anzahl der Tiere beträgt 336. Die Anzahl der Beine ist 2502. Es
  gibt doppelt so viele Schafe wie Ziegen. Es gibt 60 Hufe, die den
  Boden berühren. 
  Bestimmen Sie die jeweilige Anzahl der Tiere.
\end{question}
\begin{solution}
  Im Stall sind 21 Hühner, 10 Schafe, 5 Ziegen und 300 Spinnen (mit je 8 Beinen).
\end{solution}
\begin{question}
  Für die Zubereitung eines Multivitaminsaftgetränks werden
  Ananassaft, Pampelmusensaft und Mangosaft benötigt. Der Fruchtanteil
  des Ananassafts A betrage 40\,\%. Der Pampelmusensaft P enthält
  30\,\% Fruchtanteil, der Mangosaft M 60\,\% Fruchtanteil. Durch die
  Mischung soll 1 Liter Fruchtsaft entstehen, dessen Fruchtanteil
  50\,\% beträgt, wobei der Anteil an Pampelmusen- und Ananassaft
  gleich hoch sein sollen.
  Bestimmen Sie die Anteile von A, P und M bei einer
  möglichen Mischung.\footnote{ \url{http://lernportal.ziemke-koeln.de/mathematik/gk11/uebungen/algebra/aufg_lgs5.htm}}
\end{question}
\begin{solution}
  Die Mischung besteht aus 0,6\,l Mango, 0,2\,l Pampelmuse und 0,2\,l Ananas.
\end{solution}
\begin{question}
  Ein Weinhändler bietet vier Sortimente mit je zehn Flaschen und drei Sorten an:
Sortiment Luxus besteht aus zwei Flaschen der Sorte A, drei Flaschen
der Sorte B und fünf Flaschen der Sorte C. Es kostet 74,50\,\EUR . 
Sortiment Genießer besteht aus drei Flaschen der Sorte A, vier
Flaschen von B und drei Flaschen von C. Es kostet 66,00\,\EUR .
Sortiment Standard besteht aus drei Flaschen der Sorte A, fünf der
Sorte B und zwei der Sorte C und kostet 62,50\,\EUR .
Sortiment Einsteiger besteht aus fünf Flaschen der Sorte A, drei
Flaschen von B und zwei Flaschen von C. Es kostet 59,50\,\EUR .
Ermitteln Sie die Preise der einzelnen Flasche jeder Sorte.\footnote{\url{http://lernportal.ziemke-koeln.de/mathematik/gk11/uebungen/algebra/aufg_lgs5.htm}}

\end{question}
 \begin{solution}
  Die Weinflaschen der Sorte A kosten 4,50\,\EUR , Flaschen der Sorte
  B kosten 6,00\,\EUR , und für Sorte C sind stolze 9,50\,\EUR\ zu zahlen. 
 \end{solution}
% \section{Matrix-Vektor-Schreibweise}
\newpage
\section{Lösungen der Aufgaben}
{\scriptsize\printsolutions}

\end{document}


